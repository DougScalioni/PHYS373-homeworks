\documentclass[answers]{exam}\newcommand{\repositoryInformationSetup}{     \usepackage[dvipsnames]{xcolor}     \usepackage[ angle=90, color=black, opacity=1, scale=2, ]{background}      \SetBgPosition{current page.west}      \SetBgVshift{-4.5mm}      \backgroundsetup{contents={{\color{green}\texttt{-{}-} differs from commit \texttt{f3526e2} in 0 files}}} } \newcommand{\commit}{{{\color{green}f3526e2}}}\usepackage{amsmath}
\usepackage{xspace}
\usepackage{bbm}
\usepackage{ifthen}



\newcommand{\repoURL}{https://github.com/evanberkowitz/umd-phys-373}



\newcommand{\secref}[1]{Sec.~\ref{sec:#1}}
\newcommand{\Secref}[1]{Section~\ref{sec:#1}}
\newcommand{\appref}[1]{App.~\ref{sec:#1}}
\newcommand{\Appref}[1]{Appendix~\ref{sec:#1}}
\newcommand{\tabref}[1]{Tab.~\ref{tab:#1}\xspace}
\newcommand{\Tabref}[1]{Table~\ref{tab:#1}\xspace}
\newcommand{\figref}[1]{Fig.~\ref{fig:#1}\xspace}
\newcommand{\Figref}[1]{Figure~\ref{fig:#1}\xspace}
\newcommand{\Eqref}[1]{Equation~\ref{eq:#1}\xspace}
\def\Ref#1{Ref.~\cite{#1}} \newcommand{\Reference}[1]{Reference~\cite{#1}}
\newcommand{\Refs}[1]{Refs.~\cite{#1}}
\newcommand{\References}[1]{References~\cite{#1}}



\newcommand{\issue}[1]{\href{\repoURL/issues/#1}{Issue #1}}
\newcommand{\pullrequest}[1]{\href{\repoURL/pulls/#1}{Pull Request #1}}



\newcommand{\arxiv}[1]{\href{http://www.arxiv.org/abs/#1}{arXiv:#1}}



\newcommand{\goesto}{\ensuremath{\rightarrow}}
\newcommand{\infinity}{\infty}
\newcommand{\Integers}{\mathbb{Z}\xspace}
\newcommand{\integers}{\Integers}
\newcommand{\one}{\ensuremath{\mathbbm{1}}}
\newcommand{\order}[1]{\ensuremath{\mathcal{O}\left(#1\right)}\xspace}
\newcommand{\Rationals}{\mathbb{Q}\xspace}
\newcommand{\Reals}{\mathbb{R}\xspace}
\newcommand{\Complexes}{\mathbb{C}\xspace}
\newcommand{\union}{\ensuremath{\cup}}
\DeclareMathOperator{\erf}{erf}
\newcommand{\laplace}[1]{\ensuremath{\mathcal{L}\left\{#1\right\}}\xspace}
\newcommand{\inverselaplace}[1]{\ensuremath{\mathcal{L}\inverse\left\{#1\right\}}\xspace}


\DeclareMathOperator{\odd}{odd}
\DeclareMathOperator{\even}{even}
\DeclareMathOperator{\sinc}{sinc}
\DeclareMathOperator{\real}{Re}
\DeclareMathOperator{\imag}{Im}





\DeclareMathOperator{\sech}{sech}
\DeclareMathOperator{\csch}{csch}
\DeclareMathOperator{\arccosh}{arccosh}
\DeclareMathOperator{\arcsinh}{arcsinh}
\DeclareMathOperator{\arctanh}{arctanh}
\DeclareMathOperator{\arcsech}{arcsech}
\DeclareMathOperator{\arccsch}{arccsch}
\DeclareMathOperator{\arccoth}{arccoth}



\DeclareMathOperator{\arcsec}{arcsec}
\DeclareMathOperator{\arccot}{arccot}
\DeclareMathOperator{\arccsc}{arccsc}



\newcommand{\oneover}[1]{\ensuremath{\frac{1}{#1}}}                             \newcommand{\inverse}{\ensuremath{^{-1}}}                                       \providecommand{\half}{\ensuremath{\frac{1}{2}} }                               \renewcommand{\half}{\ensuremath{\frac{1}{2}} }                                 \newcommand{\quarter}{\ensuremath{\frac{1}{4}} }                                



\newcommand{\dd}[3][1]{
    \ifthenelse { \equal {#1} {1} }
                {\ensuremath{\frac{d#2}{d#3}}}
                {\ensuremath{\frac{d^{#1}#2}{d#3^{#1}}}}
    }

\newcommand{\pp}[3][1]{
    \ifthenelse { \equal {#1} {1} }
                {\ensuremath{\frac{\partial#2}{\partial#3}}}
                {\ensuremath{\frac{\partial^{#1}#2}{\partial#3^{#1}}}}
    }

\newcommand{\ppp}[3]{\ensuremath{\frac{\partial^2#1}{\partial#2\,\partial#3}}}

\newcommand{\grad}{\ensuremath{\nabla}\xspace}
\newcommand{\laplacian}{\ensuremath{\grad^2}\xspace}

\providecommand{\id}{}
\renewcommand{\id}[1]{\ensuremath{\; \mathrm{d}#1}}

\newcommand{\abs}[1]{\ensuremath{\left| #1 \right|}\xspace}
\newcommand{\magnitude}{\abs}
\newcommand{\average}[1]{\ensuremath{\left\langle #1 \right\rangle}\xspace}

\newcommand{\ket}[1]{\ensuremath{\left|\;#1\;\right\rangle}}
\newcommand{\bra}[1]{\ensuremath{\left\langle\;#1\;\right|}}
\newcommand{\bracket}[2]{\ensuremath{\left\langle\;#1\;\middle|\;#2\;\right\rangle}}
\let\braket\bracket
\newcommand{\operator}[3]{\ensuremath{\left|\;#1\;\middle\rangle\; #2\; \middle\langle\;#3\;\right|}}



\newcommand{\identity}{\ensuremath{\mathds{1}}}
\newcommand{\diag}[1]{\ensuremath{\text{diag}\left(#1\right)}}
\newcommand{\tr}[1]{\ensuremath{\text{tr}\left[#1\right]}}
\newcommand{\transpose}{\ensuremath{{}^{\top}}}
\newcommand{\adjoint}{\ensuremath{{}^{\dagger}}}








\newcommand{\bash}{\texttt{bash}\xspace}
\newcommand{\git}{\texttt{git}\xspace}
\newcommand{\make}{\texttt{make}\xspace}
\newcommand{\mpi}{\texttt{MPI}\xspace}
\newcommand{\python}{\texttt{python}\xspace}

\let\builtinLaTeX\LaTeX
\def\LaTeX{\builtinLaTeX\xspace}
 \usepackage{amsmath,amssymb}
\usepackage{bm}
\usepackage{comment}
\usepackage{graphicx}
\usepackage[dvipsnames]{xcolor}
\usepackage{tikz}
\usepackage{tkz-euclide}
\usepackage{slashed}
\usepackage[
    colorlinks=true,
    allcolors=blue
]{hyperref}
\usepackage{tikz}
\usetikzlibrary{calc,patterns,decorations.pathmorphing,decorations.markings}
\usepackage{pgfplots}




\providecommand{\repositoryInformationSetup}{} \repositoryInformationSetup








\begin{document}

\title{Homework 12 --- PHYS373 2021}

\author{Dr.~Evan Berkowitz	\\
\href{mailto:evanb@umd.edu}{evanb@umd.edu}}

\date{Due May 13, 2021}

\maketitle

\begin{questions}

	\section*{PDEs}
	\question Boas 13.1.2a

\begin{solution}\end{solution}
 	\question Boas 13.2.3

\begin{solution}\end{solution}
 	\question Boas 13.6.3.  Notice that the frequencies $\nu_{nm}$ are the \emph{linear} frequences.  I used the \emph{angular} frequencies, and found
\begin{equation}
	\omega_{nm} = \pi v \sqrt{\left(\frac{n}{a}\right)^2 + \left(\frac{m}{b}\right)^2}
\end{equation}
since the linear frequency $\nu = \omega/2\pi$ in terms of the angular frequency $\omega$.

Plot the modes at $t=0$ with $(m,n)=(1,1), (1,2), (2,1)$ and $(2,2)$.

Please enjoy some YouTube videos of Chladni's figures.  I think the best one might be \href{https://www.youtube.com/watch?v=OLNFrxgMJ6E}{this one from the Royal Institution} but there's also a great one from \href{https://www.youtube.com/watch?v=dTReFclu_PU}{Sixty Symbols}; just google around!  Caution: don't blow out your ears!

\begin{solution}\end{solution}
 

	\clearpage
	\section*{As Always}
	\question How long did this problem set take you?
	
	\section*{Rad YouTube Videos}
	Don't write anything: Just enjoy them!  Maybe as a study break?

	\question Check out \href{https://www.youtube.com/watch?v=dihQuwrf9yQ}{this phenomenal piece of pyrotechnics}. See if you can figure out how to compute the wavelengths of the flame height!
	\question You should now have a quantitative understanding of how to solve the problem that's the topic of \href{https://www.youtube.com/watch?v=ToIXSwZ1pJU}{this 3Blue1Brown video}.

	\section*{Optional Practice}

	\question 13.3.7
	\question 13.3.10, 13.3.11
	\subsection*{Spherical Harmonics}
	\question In the last homework we prepared to understand the Laplacian in spherical coordinates.
Recall that
\begin{align*}
	r^2 &= x^2 + y^2 + z^2			&	x &= r \sin\theta \cos\phi
	\\
	\tan \phi &= \frac{y}{x}		&	y &= r \sin\theta \sin\phi
	\\
	\cos \theta &= \frac{z}{r}		&	z &= r \cos\theta
\end{align*}
and that the \emph{Laplacian}
\begin{equation}
	\grad^2 f(x,y,z) = \left(\partial_x^2 + \partial_y^2 + \partial_z^2\right )f = \left(\oneover{r^2} \partial_r r^2 \partial_r + \oneover{r^2\sin\theta} \partial_\theta \sin\theta \partial_\theta + \oneover{r^2\sin^2\theta} \partial^2_\phi\right) f
\end{equation}
It's quite a slog to do all the algebra to show that form of the Laplacian; it works similarly to the cylindrical case, which we will review in class.

Consider the \emph{Helmholtz} equation $(\grad^2 + k^2)f=0$.
Recall that on HW11 you showed that $\int \id{V}$ where we integrated over all space was $\int_0^\infty r^2 \id{r} \int_{-1}^{+1}{\rm d}(\cos\theta) \int_0^{2\pi} \id{\phi}$.  We hope to find basis functions that are orthogonal under the inner product
\begin{equation}
	\braket{f}{g} = \int_{0}^\infty r^2 \id{r} \int_{-1}^{+1} {\rm d}(\cos\theta) \int_0^{2\pi} \id{\phi}\; f^*(r,\theta,\phi) g(r,\theta,\phi).
\end{equation}
\begin{parts}
	\part Guess $f(r,\theta,\phi) = R(r) Y(\theta,\phi)$ and separate variables to find
	\begin{align}
		\label{eq:separate R and Y}
		\partial_r r^2 \partial_r R + k^2 r^2 R &= \Lambda R
		&
		\oneover{\sin\theta}\partial_\theta \sin\theta \partial_\theta Y + \oneover{\sin^2\theta}\partial^2_\phi Y &= -\Lambda Y
	\end{align}
	for some separation constant $\Lambda$.
	\begin{solution}\end{solution}
	\part Actually, we can further separate the angular part.  Let $Y=\Theta(\theta)\Phi(\phi)$ and separate to find
	\begin{align}
		\label{eq:angular equations}
		\partial_\phi^2 \Phi &= - \lambda^2 \Phi
		&
		\sin\theta \partial_\theta \sin\theta \partial_\theta \Theta + \Lambda (\sin^2 \theta) \Theta &= \lambda^2 \Theta
	\end{align}
	for some separation constant $\lambda^2$.

	\part $e^{\pm i\lambda \phi}$ solves the equation for $\Phi$.  However, recall that $\phi$ is $2\pi$-periodic.  Show that this implies $\lambda$ must be an integer.  It is canonical to call this integer $m$.
	\begin{solution}\end{solution}

	\part Recall that on HW04Q7 you showed that for integers $m$ and $n$, $\int_0^{2\pi} \id{\phi}\; e^{i(m-n)\phi} = 2\pi \delta_{mn}$ and that on HW11Q you showed that to integrate over all space in spherical coordinates, we had to integrate $\phi$ from 0 to $2\pi$.  Suppose we had two functions $u=R\Theta\Phi_m$ and $v=\tilde{R}\tilde{\Theta}\Phi_n$  for possibly-the-same/possibly-different ($R$ and $\tilde{R}$) and ($\Theta$ and $\tilde{\Theta}$) and $\Phi_m=e^{im\phi}$ and $\Phi_n=e^{in\phi}$.  Prove that when integrated over all space $\int \id{V}\; u^* v$ must be proportional to $\delta_{mn}$.
	\begin{solution}\end{solution}


	\part Knowing that $\lambda=m$ must be an integer, show that the $\Theta$ equation in \eqref{eq:angular equations} can be written
	\begin{equation}
		\oneover{\sin\theta} \partial_\theta \sin\theta \partial_\theta \Theta - \frac{m^2}{\sin^2\theta} \Theta = -\Lambda \Theta
	\end{equation}
	\begin{solution}\end{solution}
	This equation is a form of the \href{https://en.wikipedia.org/wiki/Associated_Legendre_polynomials}{\emph{Associated Legendre Equation}} (see also Boas 12.10).
	If we demand that $\Theta$ is \emph{regular} (meaning that it does not blow up or have any sharp cusps) then the eigenfunctions are the \emph{associated Legendre polynomials} $P_\ell^m(\cos\theta)$ with eigenvalues $\Lambda = \ell(\ell+1)$ with integer $\ell\geq0$ and $-\ell\leq m \leq +\ell$ (notice that there's a different equation for each $m$, so the eigenfunctions labelled by $m$ satisfy the equation with that particular value of $m$).
	The solutions $P_\ell^m(\cos\theta)$ satisfy the \emph{orthogonality relation}
	\begin{equation}
		\label{eq:associated legendre orthogonality}
		\int_{-1}^{+1} P_{\ell'}^m P_\ell^m \id{x} = \frac{2(\ell+m)!}{(2\ell+1)(\ell-m)!} \delta_{\ell',\ell}
	\end{equation}
	(If $m$ are different they might not be orthogonal, but in that case the $\phi$ dependence will do the work and ensure orthogonality).

	Show that the \emph{spherical harmonics}
	\begin{equation}
		Y_\ell^m(\theta,\phi) = (-1)^m \sqrt{\frac{2\ell+1}{4\pi} \cdot \frac{(\ell-m)!}{(\ell+m)!}} P_\ell^m(\cos\theta) e^{i m \phi},
	\end{equation}
	satisfy
	\begin{equation}
		\int_0^{2\pi} \id{\phi} \int_{-1}^{+1} {\rm d}(\cos\theta)\; Y_{\ell'}^{m'}(\theta,\phi)^*\; Y_{\ell}^{m}(\theta,\phi) = 
		\delta_{\ell',\ell}\delta^{m',m}
	\end{equation}
	(We write the indices of the second Kronecker delta upstairs to match the convention of where we write the $m$ index on $Y$; but it means the same thing as if the indices are downstairs).
	\begin{solution}\end{solution}



	\part Things are really coming together!  By demanding nice, smooth functions we have fixed $\ell$ and $m$ to be integers, $-\ell\leq m \leq +\ell$, and we determined the separation constants $\lambda=m$ and $\Lambda=\ell(\ell+1)$, .
	We're left with just the separation equation for $R$ to solve,
	\begin{equation}
		\partial_r r^2 \partial_r R + r^2 k^2 R = \ell(\ell+1) R
	\end{equation}
	which is solved by the \emph{spherical Bessel functions} $R = j_\ell(k r)$ and $y_\ell(k r)$ (there are solutions for all $k$ and each $\ell$).
	While both $j_\ell$ and $y_\ell$ oscillate and decay with increasing $r$, the $j_\ell$ functions are finite at $r=0$ while the $y_\ell$ functions explode at $r=0$.
	The spherical bessel functions satisfy the orthogonality relation
	\begin{equation}
		\label{eq:spherical bessel orthogonality}
		\int_0^{\infty} j_\ell(k'r) j_\ell(k r) r^2 \id{r} = \frac{\pi}{2k^2} \delta(k'-k).
	\end{equation}
	Again, if the two $\ell$s were different they might not be orthogonal, but in that case the $Y_{\ell'}^{m'}$ and $Y_\ell^m$ \emph{would} be orthogonal.
	If we want to describe functions that stay finite at $r=0$, we get the basis $\ket{k,\ell,m}$ which, in the $\ket{r,\theta,\phi}$ basis can be written
	\begin{align}
		\ket{k,\ell,m} &= \int_0^{\infty} r^2 \id{r} \int_{-1}^{+1} {\rm d}(\cos\theta) \int_0^{2\pi} \id{\phi}\; \sqrt{\frac{2k^2}{\pi}} j_\ell(k r)Y_\ell^m(\theta, \phi) \ket{r,\theta,\phi}	
		\\\nonumber &\qquad (m,\ell \in \Integers;\; 0\leq \ell;\; -\ell \leq m \leq +\ell;\; k\text{ real})
	\end{align}
	Show that if $\braket{r',\theta',\phi'}{r,\theta,\phi} = r^{-2} \delta(r'-r)\delta(\cos\theta'-\cos\theta)\delta(\phi'-\phi)$ (note the leading $r^{-2}$ and the cosines!) then this basis has the orthogonality relation
	\begin{equation}
		\braket{k',\ell',m'}{k,\ell,m} = \delta(k'-k)\delta_{\ell',\ell} \delta{m',m}
	\end{equation}
	(Hint: Use your orthogonality result for the spherical harmonics $Y$ and \eqref{eq:spherical bessel orthogonality})
	\begin{solution}\end{solution}
	
\end{parts}
{\bf The rest is just for your education!}
Suppose we have a vector $\ket{f} = \int r^2 \id{r} \int {\rm d}(\cos\theta) \int \id{\phi} f(r,\theta,\phi) \ket{r,\theta,\phi}$ that we want to express in terms of the basis vectors
	\begin{equation}
		\label{eq:partial wave decomposition}
		\ket{f} = \sum_{\ell=0}^{\infty} \sum_{m=-\ell}^{+\ell} \int_{-\infty}^{+\infty}\id{k}\; f_\ell^m(k) \ket{k, \ell, m}
	\end{equation}
We can compute the \emph{partial wave amplitudes} $f_\ell^m(k)$ as follows:
Take the inner product $\braket{k',\ell',m'}{f}$ in both bases.  In the spherical basis,
\begin{align*}
	\braket{k',\ell',m'}{f} 
		&= \sum_{\ell=0}^{\infty} \sum_{m=-\ell}^{+\ell} \int_{-\infty}^{+\infty}\id{k}\; f_\ell^m(k) \braket{k',\ell',m'}{k, \ell, m}
	\\	&= \sum_{\ell=0}^{\infty} \sum_{m=-\ell}^{+\ell} \int_{-\infty}^{+\infty}\id{k}\; f_\ell^m(k) \delta(k'-k)\delta_{\ell',\ell}\delta^{m',m}
	\\	&= f_{\ell'}^{m'}(k')
\end{align*}
because the delta functions pick out only the single value of interest from each sum/integral.
In the position (radial/angular) basis,
\begin{align*}
	\braket{k',\ell',m'}{f} 
		&= \int r^2 \id{r} \int {\rm d}(\cos\theta) \int \id{\phi} f(r,\theta,\phi) \braket{k',\ell',m'}{r,\theta,\phi}
	\\
	f_{\ell'}^{m'}(k')
		&= \int r^2 \id{r} \int {\rm d}(\cos\theta) \int \id{\phi} f(r,\theta,\phi) \left(\sqrt{\frac{2k'^2}{\pi}}j_\ell(k r) Y_{\ell'}^{m}(\theta,\phi)\right)^* 
\end{align*}
where all the integrals are over the whole domain.

Finally!  What is the laplacian applied to \ket{f}?  Since each basis vector satisfies $\grad^2\ket{k,\ell,m} = -k^2 \ket{k,\ell,m}$ we must have
\begin{align}
	\grad^2 \ket{f} = 
			= \sum_{\ell=0}^{\infty} \sum_{m=-\ell}^{+\ell} \int_{-\infty}^{+\infty}\id{k}\;  f_\ell^m(k) (-k^2) \ket{k, \ell, m}
\end{align}
So, in situations where a spherical geometry is more natural, we know the functions where it is easy to apply the laplacian operator!

In other words,
\begin{equation}
	\grad^2 = \sum_{\ell=0}^{\infty} \sum_{m=-\ell}^{+\ell} \int_{-\infty}^{+\infty} \id{k} \operator{k, \ell, m}{(-k^2)}{k,\ell,m}.
\end{equation}
 
\end{questions}

\end{document}
