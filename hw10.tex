\documentclass[answers]{exam}\newcommand{\repositoryInformationSetup}{     \usepackage[dvipsnames]{xcolor}     \usepackage[ angle=90, color=black, opacity=1, scale=2, ]{background}      \SetBgPosition{current page.west}      \SetBgVshift{-4.5mm}      \backgroundsetup{contents={{\color{green}\texttt{-{}-} differs from commit \texttt{40a9b87} in 0 files}}} } \newcommand{\commit}{{{\color{green}40a9b87}}}\usepackage{amsmath}
\usepackage{xspace}
\usepackage{bbm}
\usepackage{ifthen}


\newcommand{\secref}[1]{Sec.~\ref{sec:#1}}
\newcommand{\Secref}[1]{Section~\ref{sec:#1}}
\newcommand{\appref}[1]{App.~\ref{sec:#1}}
\newcommand{\Appref}[1]{Appendix~\ref{sec:#1}}
\newcommand{\tabref}[1]{Tab.~\ref{tab:#1}\xspace}
\newcommand{\Tabref}[1]{Table~\ref{tab:#1}\xspace}
\newcommand{\figref}[1]{Fig.~\ref{fig:#1}\xspace}
\newcommand{\Figref}[1]{Figure~\ref{fig:#1}\xspace}
\newcommand{\Eqref}[1]{Equation~\ref{eq:#1}\xspace}
\def\Ref#1{Ref.~\cite{#1}} \newcommand{\Reference}[1]{Reference~\cite{#1}}
\newcommand{\Refs}[1]{Refs.~\cite{#1}}
\newcommand{\References}[1]{References~\cite{#1}}



\newcommand{\issue}[1]{\href{\repoURL/issues/#1}{Issue #1}}
\newcommand{\pullrequest}[1]{\href{\repoURL/pulls/#1}{Pull Request #1}}



\newcommand{\arxiv}[1]{\href{http://www.arxiv.org/abs/#1}{arXiv:#1}}



\newcommand{\goesto}{\ensuremath{\rightarrow}}
\newcommand{\infinity}{\infty}
\newcommand{\Integers}{\mathbb{Z}\xspace}
\newcommand{\integers}{\Integers}
\newcommand{\one}{\ensuremath{\mathbbm{1}}}
\newcommand{\order}[1]{\ensuremath{\mathcal{O}\left(#1\right)}\xspace}
\newcommand{\Rationals}{\mathbb{Q}\xspace}
\newcommand{\Reals}{\mathbb{R}\xspace}
\newcommand{\Complexes}{\mathbb{C}\xspace}
\newcommand{\union}{\ensuremath{\cup}}
\DeclareMathOperator{\erf}{erf}
\newcommand{\laplace}[1]{\ensuremath{\mathcal{L}\left\{#1\right\}}\xspace}
\newcommand{\inverselaplace}[1]{\ensuremath{\mathcal{L}\inverse\left\{#1\right\}}\xspace}


\DeclareMathOperator{\odd}{odd}
\DeclareMathOperator{\even}{even}
\DeclareMathOperator{\sinc}{sinc}
\DeclareMathOperator{\real}{Re}
\DeclareMathOperator{\imag}{Im}





\DeclareMathOperator{\sech}{sech}
\DeclareMathOperator{\csch}{csch}
\DeclareMathOperator{\arccosh}{arccosh}
\DeclareMathOperator{\arcsinh}{arcsinh}
\DeclareMathOperator{\arctanh}{arctanh}
\DeclareMathOperator{\arcsech}{arcsech}
\DeclareMathOperator{\arccsch}{arccsch}
\DeclareMathOperator{\arccoth}{arccoth}



\DeclareMathOperator{\arcsec}{arcsec}
\DeclareMathOperator{\arccot}{arccot}
\DeclareMathOperator{\arccsc}{arccsc}



\newcommand{\oneover}[1]{\ensuremath{\frac{1}{#1}}}                             \newcommand{\inverse}{\ensuremath{^{-1}}}                                       \providecommand{\half}{\ensuremath{\frac{1}{2}} }                               \renewcommand{\half}{\ensuremath{\frac{1}{2}} }                                 \newcommand{\quarter}{\ensuremath{\frac{1}{4}} }                                



\newcommand{\dd}[3][1]{
    \ifthenelse { \equal {#1} {1} }
                {\ensuremath{\frac{d#2}{d#3}}}
                {\ensuremath{\frac{d^{#1}#2}{d#3^{#1}}}}
    }

\newcommand{\pp}[3][1]{
    \ifthenelse { \equal {#1} {1} }
                {\ensuremath{\frac{\partial#2}{\partial#3}}}
                {\ensuremath{\frac{\partial^{#1}#2}{\partial#3^{#1}}}}
    }

\newcommand{\ppp}[3]{\ensuremath{\frac{\partial^2#1}{\partial#2\,\partial#3}}}

\providecommand{\id}{}
\renewcommand{\id}[1]{\ensuremath{\; \mathrm{d}#1}}

\newcommand{\abs}[1]{\ensuremath{\left| #1 \right|}\xspace}
\newcommand{\magnitude}{\abs}
\newcommand{\average}[1]{\ensuremath{\left\langle #1 \right\rangle}\xspace}

\newcommand{\ket}[1]{\ensuremath{\left|\;#1\;\right\rangle}}
\newcommand{\bra}[1]{\ensuremath{\left\langle\;#1\;\right|}}
\newcommand{\bracket}[2]{\ensuremath{\left\langle\;#1\;\middle|\;#2\;\right\rangle}}
\let\braket\bracket




\newcommand{\identity}{\ensuremath{\mathds{1}}}
\newcommand{\diag}[1]{\ensuremath{\text{diag}\left(#1\right)}}
\newcommand{\tr}[1]{\ensuremath{\text{tr}\left[#1\right]}}
\newcommand{\transpose}{\ensuremath{{}^{\top}}}
\newcommand{\adjoint}{\ensuremath{{}^{\dagger}}}








\newcommand{\bash}{\texttt{bash}\xspace}
\newcommand{\git}{\texttt{git}\xspace}
\newcommand{\make}{\texttt{make}\xspace}
\newcommand{\mpi}{\texttt{MPI}\xspace}
\newcommand{\python}{\texttt{python}\xspace}

\let\builtinLaTeX\LaTeX
\def\LaTeX{\builtinLaTeX\xspace}
 \usepackage{amsmath,amssymb}
\usepackage{bm}
\usepackage{comment}
\usepackage{graphicx}
\usepackage[dvipsnames]{xcolor}
\usepackage{tikz}
\usepackage{tkz-euclide}
\usepackage{slashed}
\usepackage[
    colorlinks=true,
    allcolors=blue
]{hyperref}
\usepackage{tikz}
\usetikzlibrary{calc,patterns,decorations.pathmorphing,decorations.markings}
\usepackage{pgfplots}




\providecommand{\repositoryInformationSetup}{} \repositoryInformationSetup








\begin{document}

\title{Homework 10 --- PHYS373 2021}

\author{Domingues, Douglas}

\date{Due Apr 29, 2021}

\maketitle

\begin{questions}
	\section*{The Convolution}
	\question In class we showed, by direct integration, that $t^2*t = \frac{t^4}{12}$, and checked that $\laplace{t^2}\laplace{t} = \laplace{t^2*t} = \laplace{\frac{t^4}{12}}$.
Show, by direct integration, that $t*t^2$ is also $\frac{t^4}{12}$.
This shows one specific example of the general fact that $f*g = g*f$.

\begin{solution}
We want to compute $t \ast t^2$. By the definition of the convolution:

$$t \ast t^2 = \int_0^t \tau (t-\tau)^2 \id{\tau}$$

$$ = \int_0^t \tau (t^2 -2 t \tau + \tau ^2) \id{\tau}$$

$$ = \int_0^t  (\tau t^2 -2 t \tau^2 + \tau ^3) \id{\tau}$$

$$ = [\frac{\tau^2}{2} t^2 -2 t \frac{\tau^3}{3} + \frac{\tau ^4}{4}]_0^t$$

$$ = \frac{t^4}{2} \frac{-2 t^4}{3} + \frac{t^4}{4} = \frac{6 - 8 + 3}{12} t^4 = \frac{t^4}{12}$$

\end{solution}


 	\question Suppose a garbage truck in a small town collects some material at a rate $g(t)$ ($g$ for garbage; maybe it's tons/week, or something like that).
Let's assume that each piece of that material decays according to the function $d(t)$ ($d$ for decay).
That is, if you start with 1 ton at time $t=0$ and don't add any, at a later time you have $d(t)$ tons of that material ($d$ is typically a decreasing function, and $d(0)=1$ because after no time the material hasn't decayed at all).

Let's assume the garbage dump starts empty and the garbage truck arrives at the dump once a week and deposits $g(t) \Delta t$ tons of material ($\Delta t$=1 week).  
After the zeroeth week the total in the dump is $(g(0) \Delta t) d(0)$.
After the first week, the total is composed of what was already in the dump, but now it's had one week to decay ($(g(0) \Delta t) d(1)$; plus the new material is added, $(g(1) \Delta t) d(0)$.

\begin{parts}
\part How much material is in the dump after the truck deposit's the second week's haul?

\begin{solution}
$(g(2) \Delta t) d(0) + (g(1) \Delta t) d(1) + (g(0) \Delta t) d(2)$
\end{solution}

\part How much after the third week's haul?

\begin{solution}
$(g(3) \Delta t) d(0) + (g(2) \Delta t) d(1) + (g(1) \Delta t) d(2) + (g(0) \Delta t) d(3)$
\end{solution}

\part You don't have to write anything for this part, but convince yourself that after $n$ weeks the dump contains
\begin{equation}
	\text{total material} = \sum_{w=0}^{n} g(w) d(n-w) \Delta t
\end{equation}
(you can use this to check you answer to the previous two parts).

\begin{solution}
Makes sense. Here's the sum for week 3:

$$ \text{total material} = \sum_{w=0}^{3} g(w) d(3-w) \Delta t $$
$$ =  (g(0) \Delta t) d(3) + (g(1) \Delta t) d(2) + (g(2) \Delta t) d(1) + (g(3) \Delta t) d(0)$$ 
\end{solution}

Suppose instead of a small town with a garbage truck, your dump services the New York Department of Sanitation.  So, instead of garbage arriving once a week, garbage arrives at the dump continuously at a rate $g(t)$.  Convince yourself (you don't have to write anything as an answer) that the Riemann sum from the previous part goes to
\begin{equation}
	\int_{0}^{t} g(\tau) d(t-\tau) \id{\tau}
\end{equation}
where we took the limit $\Delta t\goesto 0$ and changed $w$ (named to indicate weeks) to $\tau$ to indicate a continuous time.

\part Suppose you work for a normal dump and it collects styrofoam, at a rate $g(t)$.  To a good approximation, styrofoam never decays, $d(t)=1$, independent of $t$.  How much styrofoam is in the dump at time $t$? (Leave your answer as an integral.  Does the answer make sense?)

\begin{solution}
  $$\int_{0}^{t} g(\tau) d(t-\tau) \id{\tau}$$

Since $d=1 \forall \tau$ 
 
  $$\int_{0}^{t} g(\tau) \id{\tau}$$
  
  
Yes, the answer makes sense. Afterall, the amount of styrofoam in the dump is what was put there, or the accumulation of what was continuously deposited there over time.
\end{solution}

\end{parts}

\question Suppose you work for the Department of Energy and run a nuclear waste storage facility in a deep, geologically stable underground salt cave.
The facility opens in 1990 and collects waste material at a constant rate and is sealed in 2030, at which point the facility is closed and no new waste may be accepted; $g(t) = g u_{1990,2030}(t)$ ($g$ has units of mass/time, say kg/year).

The radioactive material decays with a halflife $h$; $d=e^{-t \ln 2/h}$.
We can show that the amount of radioactive material in the repository is given by
\begin{equation}
g*d = \frac{gh}{\ln 2} \left[u(t-1990)\left(1-e^{-(t-1990) \ln 2/h} \right)- u(t-2030)\left(1-e^{-(t-2030) \ln 2 / h}\right)\right].
\end{equation}
in two ways.

First, we can evaluate a very tricky integral as follows:
We convolve
\begin{align*}
	g*d 
		&=	\int_{0}^{t} g(\tau) e^{-(t-\tau) \ln 2/h} \id{\tau}
	\\	&=	\int_{0}^{t} g u_{1990,2030}(\tau) e^{-(t-\tau) \ln 2/h} \id{\tau}
	\\	&=	g \int_{0}^{t} \left((u(\tau-1990) - u(\tau-2030)\right) e^{-(t-\tau) \ln 2/h} \id{\tau}
	\\	&=	g e^{-t \ln 2/h} \int_{0}^{t} (u(\tau-1990) - u(\tau-2030) e^{\tau \ln 2/h} \id{\tau}
	\\	&=	g e^{-t \ln 2/h} \left[\int_{0}^{t} u(\tau-1990) e^{\tau \ln 2/h} \id{\tau} - \int_{0}^{t} u(\tau-2030) e^{\tau \ln 2/h} \id{\tau}\right]
\end{align*}
Now we need to think.  The first integral is zero if $t<1990$, because the integrand is zero there.  So that piece should be proprotional to $u(t-1990)$.
Similarly the second integral is proportional to $u(t-2030)$.
\begin{align*}
	g*d
		&=	g e^{-t \ln 2/h} \left[ u(t-1990)\int_{1990}^t e^{\tau \ln 2/h} \id{\tau} - u(t-2030)\int_{2030}^{t} e^{\tau \ln 2/h} \id{\tau}\right]
	\\	&=	g e^{-t \ln 2/h} \left[ u(t-1990) \left(\frac{e^{t \ln2/h}-e^{1990 \ln 2/h}}{\ln 2/h}\right) - u(t-2030) \left(\frac{e^{t \ln 2/h} - e^{2030 \ln 2/h}}{\ln 2 / h}\right)\right]
	\\	&=	\frac{gh}{\ln 2} \left[u(t-1990)\left(1-e^{-(t-1990) \ln 2/h} \right)- u(t-2030)\left(1-e^{-(t-2030) \ln 2 / h}\right)\right]
\end{align*}
{\bf Arrive at this same conclusion by computing the convolution via the Laplace transform: $g*d = \inverselaplace{\laplace{g} \laplace{d}}$.} (You'll need to perform partial fractions and use some of the rules we've shown in class; you can look them up in the table in Boas.)
\begin{solution}\end{solution}

\question {\bf This question does not constitute financial advice!}\footnote{
Seriously, this is a cartoon.
I'm making all sorts of simplifying assumptions!
For example, it's false that the market grows 10\% a year; that's a rough historical average that doesn't necessarily indicated future growth.
Talk to someone who knows more about this than a physics professor when planning for your retirement.
Still---the lessons of this problem:
`in general the earlier you start saving for retirement, the better'
and
`those first few years can make an enormous difference later down the line' are true.
}
At age $A$ you start maximizing your contributions to your \href{https://en.wikipedia.org/wiki/401(k)}{401(k)}, an individual retirement saving account that invests in the stock market.
Each year until you retire at age R you deposit into your account the maximum allowed by law, $M=\$19,500$ per year, and you never make a withdrawal\footnote{
Two things:
1) the IRS increases $M$ now and then to keep up with inflation, which we're ignoring in this question; it's a few percent a year which is really nontrivial!
2) Retirement accounts generally punish withdrawals before retirement age.
}.
So, your deposit rate is $d(t) = M u_{A,R}(t)$
Suppose your 401(k) grows according to the historical market return, 10\% per year.
In other words, $g(t) = e^{rt}$ where $r\approx 0.095$ is set by $g(1) = 1.1$.
Accounting for your deposits and the growth in your portfolio, the total balance $b$ in your account as a function of time is $b(t) = d*g$.

\begin{parts}
\part Argue, using the solution to the nuclear dumping question, or by using Laplace transforms, that
\begin{equation}
	b(t) = d*g = \frac{M}{r} \left[u(t-A)\left(e^{r(t-A)}-1\right) - u(t-R)\left(e^{r(t-R)} - 1\right)\right]
\end{equation}
\begin{solution}\end{solution}

\part Suppose you retire at $R=65$.  Plot how much money you will have in your 401(k) at $t=R=65$ as a function of $A$ from 25 to 65, the year you start making contributions.

\begin{solution}\end{solution}

\part Obviously, the total quantity of money you have deposited at time $t$ is $M(t-A)$.
Plot $b(R) / M(R-A)$ as a function of $A$ the age you start saving, assuming you retire at $R=65$.
This ratio is your `bang-for-the-buck'.

\begin{solution}\end{solution}

\part According to this naive model, how much more money will you save if you start saving at $A=25$ than if you start at $A=30$, assuming $R=65$?

\begin{solution}\end{solution}

\end{parts}
Reality check: the average American---especially the average 25-year-old---doesn't typically have a job where they can easily deposit the maximum $M$ into their 401(k)---it'd be too large a percentage of their take-home pay.
So what happens more realistically is that the deposit rate $d$ starts out low and grows as you advance in your career.
Don't feel stressed out if you don't get such a lucrative job at 25; the point of this question is to prepare you to understand the gist of the computation you must do to roughly understand your future; you can plug in more realistic functions!


 
	\section*{Solving ODEs with Laplace Transforms (using convolutions!)}
	\question Boas 8.12.5

\begin{solution}\end{solution}
 	\question Consider a system goverened by the ODE
\begin{equation}
	\ddot{y} + 2b \dot{y} + \omega_0^2 y = f(t)
\end{equation}
which starts with $\dot{y}(0)=\dot{y}_0$ and $y(0)=y_0$.

\begin{parts}
\part Show that the Laplace transform of the ODE leads to
\begin{equation}
	Y
	=
	\frac{F}{s^2+2bs+\omega_0^2} + \frac{s y_0}{s^2+2bs+\omega_0^2} + \frac{2 b y_0 + \dot{y}_0}{s^2+2bs+\omega_0^2}
\end{equation}
where $Y=\laplace{y}$ and $F=\laplace{f}$.
\begin{solution}\end{solution}
Let's stop and take this this form in for a second.  The first piece is the same no matter what the initial conditions are---it therefore corresponds to a particular solution.  The other two pieces have constants in them fixed by the initial conditions---they are part of the complementary function (remember?  the solution to the associated homogeneous equation?)

\part Let's first consider $F=0$ with the above initial conditions so that we get the appropriate complementary function.  Assume were are not in the critically damped case.  Show that
	\begin{align}
		y_c &= \oneover{2\sqrt{b^2-\omega_0^2}} \left[ \left(\dot{y}_0 - y_0r_-\right) e^{r_+ t} - \left(\dot{y}_0 - y_0r_+\right) e^{r_- t}\right] u(t)
		&
		r_\pm &= -b \pm \sqrt{b^2-\omega_0^2}
	\end{align}
	where $r_\pm$ are the charactersitic roots of the equation (remember?!).
	Notice that this solution has the property that $y(0) = y_0$ and $\dot{y}(0) = \dot{y}_0$ it satisfied the ODE for all $t>0$; where we expect Laplace transform methods to work.
\begin{solution}\end{solution}

	\part Show that
	\begin{equation}
		W(t) = \inverselaplace{\oneover{s^2+2bs+\omega_0^2}} = \oneover{2\sqrt{b^2-\omega_0^2}} \left[ e^{r_+ t} - e^{r_- t}\right] u(t),
	\end{equation}
	again assuming we're not in the critical case.  (You can probably re-use some of the partial-fractions from the previous part, if that's how you did it.)

	\begin{solution}\end{solution}

	\part Now you know the general solution to the ODE for any $f(t)$---just convolve $W*f$ to find the particular solution and add the complementary solution.
	Try it for a delta-function impulse at time $t_i>0$ $f = f_i \delta(t-t_i)$.
	Show that the particular solution
	\begin{equation}
		y_p = W*f = \frac{e^{r_+-(t-t_i)} - e^{r_-(t-t_i)}}{r_+-r_-} f_i u(t-t_i)
	\end{equation}
	by directly evaluating the convolution integral $W*f = \int_0^{t} \id{\tau}\; W(\tau) f(t-\tau)$.
	You might want to consider two cases: $t<t_i$ and $t>t_i$ separately, as shown for the question on nuclear dumping.
	Notice that the particular solution only `kicks in' once the system receives the impulse---that makes physical sense!
	Also---what is the behavior?
	It's exactly what you expect---decaying exponentially (there are decaying oscillations if we're in the underdamped case).
	\begin{solution}\end{solution}

\end{parts}
In class we mentioned that the $\delta$ is the identity element for the convolution operation, $\delta*g = g*\delta = g$ for any $g$.  We also mentioned that convolution is associative, $(a*b)*c = a*(b*c)$, because the Laplace transform of both sides is $ABC$.
Suppose we have some other signal, not $f(t) = \delta(t)$ (setting $t_i=0$ and $f_i=1$) but $g(t)$.  The particular solution given $f$ is $f*W = \delta*W = y_p$.  Since $g= g*\delta$ the particular solution for $g$ as input is $(g*\delta)*W = g*(\delta*W) = g*y_p$ where $y_p$ is what we just found (with $t_i=0$ and $f_i=1$ plugged in---which is $W$ itself!).
The whole thing hangs together!
 	\section*{As Always}
	\question How long did this problem set take you?
	
	\section*{Optional Practice}

	\question Boas 8.10.1
	\question Boas 8.10.2
	\question Boas 8.10.{3-12} are good practice for using the table or for applying partial fractions decomposition.
	\question Boas 8.10.13
	\question Boas 8.10.{14,15} for practice solving differential equations.
	\question Boas 8.10.17; you can also try it with $+a^2$ instead of $-a^2$.
\end{questions}

\end{document}
